\section*{Definitions}

Definitions are more detailed explanations of what words in the statutes are supposed to mean. The purpose of this section is to avoid misunderstanding because of translations from Swedish (of other languages) association terminology. The definitions are not part of the statutes but accepted interpretations of words in the statutes. 

\subsection*{Non-profit}
This is a translation of the Swedish word “ideell”.

\subsection*{Motions}
Suggestions from members that are decided about in the meeting. All members have the right to send motions to the board before the meeting and the suggestions have to be brought up during the meeting (“motioner” in Swedish). 

\subsection*{Bills}
The same as motions except the suggestion is from the board (“propositioner” in Swedish). Note that individual board members can send motions as well; bills are suggestions from the entire board.

\subsection*{AOB}
Any other Business 

\subsection*{General meeting}
The general meeting is the highest decision making organ of the association, the board functions on their behalf. 

\subsection*{Discharge of liability}
During the spring meeting the association always votes about this. Discharge of liability means that the board is no longer responsible for mistakes they made during the year. If it is denied by the annual meeting, the members of the association have one year to press charges against the board (“Frågan om ansvarsfrihet” in Swedish).

\subsection*{Unanimous decision}
That no one objects the decision.