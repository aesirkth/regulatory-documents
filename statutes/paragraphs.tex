
\section*{About the Association}

\subsection{Name and location}
\begin{enumerate}
  \item The name of the association is “Association of Engineering Students in Rocketry”, abbreviated ÆSIR\footnote{The Æ letter is written, if possible, using the unicode character \textbf{U+00C6} or the HTML entity \textbf{\&\#198;}.} or AESIR.
  \item In legal matters, the non-abbreviated full name is used.
  \item The association is located in Stockholms kommun.
\end{enumerate}

\subsection{Principles}
\begin{enumerate}
  \item The association is a non-profit association.
  \item The association is politically and religiously impartial.
  \item The association works according to democratic principles.
  \item The core values of the association are \emph{collaboration}, \emph{education} and \emph{transparency}.
\end{enumerate}


\subsection{Year}
\begin{enumerate}
  \item The operational year runs from January 1st to December 31st.
  \item The fiscal year runs from January 1st to December 31st.
\end{enumerate}


\subsection{Authority}
\begin{enumerate}
  \item The general meeting is the highest governing body of the association.
  \item The board is the highest executive body of the association.
  \item Between general meetings, the board acts as the highest governing body of the association. 
\end{enumerate}

\section*{Purpose}

\subsection{Goals}
The association aims to:
\begin{enumerate}
  \item Design, build and launch rockets.
  \item Give members real world engineering experience in complex projects.
  \item Educate students in rocketry and engineering practices.
  \item Accumulate and maintain knowledge on rocketry and engineering practices.
\end{enumerate}

\subsection{Process}
The association seeks to achieve its objectives by: 
\begin{enumerate}
  \item Setting up projects to design, construct and launch rockets.
  \item Facilitating the construction and launching of rockets in legal and logistical matters.
  \item Providing a framework where members can educate themselves in engineering.
  \item Facilitating the arrangement of lectures, activities and company visits.
\end{enumerate}

\section*{Regulatory Documents}
\subsection{Availability}
The association seeks to achieve its objectives by: 
\begin{enumerate}
  \item Copies of the current version of all regulatory documents should be stored at a location – physical or digital – such that all members of the association can access them.
  \item Signed originals of regulatory documents should be stored at a safe location by the board.
\end{enumerate}

\subsection{Interpretation}
\begin{enumerate}
  \item If there are any uncertainties with the interpretation of the statutes, they are interpreted by the general meeting.
  \item Outside of the general meeting, the statutes are interpreted by the board.
  \item An interpretation of the statutes outside a general meeting must be tested at the next general meeting.
\end{enumerate}

\subsection{Statutes}
\begin{enumerate}
  \item The statutes can be changed by decision of two consecutive general meetings, of which one is the spring or fall meeting. A minimum of four (4) weeks must elapse between the two meetings.
  \item In order to change the statutes, the request to do so and the proposed changes must be included in the invitation and the meeting documents for both of the meetings required to complete the process.   
\end{enumerate}

\subsection{House rules}
\begin{enumerate}
  \item The house rules regulate the operation of the association in more detail than the statutes. 
  \item The house rules are always subordinated to the statutes.
  \item The house rules can be changed by decision of the general meeting, as long as the request to do so and a document of the proposed changes is properly sent as specified in \ref{sec:regMeetingDocs}.
\end{enumerate}

\section*{Membership}
\subsection{Definition}
\begin{enumerate}
  \item A member is someone who has signed the membership contract as well as paid their membership fee.
\end{enumerate}

\subsection{Rights}
\begin{enumerate}
  \item Members have the right to: 
  \begin{enumerate}
    \item Attend, speak and vote during general meetings.
    \item Attend board meetings.
    \item Have access to current versions of the regulatory documents of the association.
    \item Have access to attested protocols from board meetings and general meetings.
  \end{enumerate}
\end{enumerate}

\subsection{Responsibilities}
\begin{enumerate}
  \item Members have to pay the membership fee determined by the fall meeting.
\end{enumerate}

\setcounter{subsection}{14}
\subsection{Termination of membership}
Membership is terminated:
\begin{enumerate}
  \item Automatically, if the member has not paid their membership fee before January 31st.
  \item If the member requests it to the board.
  \item The general meeting can expel a member given a two-thirds majority vote. Repeals can only be made through decision by the general meeting.
\end{enumerate}

\section*{Board}
\subsection{Responsibilities}
The board is responsible for:
\begin{enumerate}
  \item The operation of the association.
  \item Making sure that decisions made during general meetings are executed.
  \item Summoning the fall and spring meetings. 
  \item Maintaining the finances of the association. 
\end{enumerate}

\subsection{Board miscellaneous}
\begin{enumerate}
  \item The board consists of at least three (3) and at most eight (8) members and should at least have a president, a treasurer and a secretary.
  \item The members of the board are individually chosen in function by the fall meeting.
  \item The board is changed January 1st, but elected on the fall meeting
  \item If there are less than three members of the board an extra general meeting needs to be held to fill the vacant position.
  \item Vacant board positions can be filled at the spring meeting.
  
\end{enumerate}

\subsection{Board meetings}
\begin{enumerate}
  \item The board has to meet at least once every quarter of the year, or if at least a third of the board members have made a written request to the president of the association.
  \item The president of the association should notify the board members about the meeting at least four (4) days in advance.
  \item The board can only make decisions if at least half of the board members are present and under the condition that the meeting is summoned in the correct way.
  \item Decisions during the board's regular meetings are made by simple majority.
  \item The board has to write decision protocols at its meetings.
  \item Copies of attested protocols should be accessible to all association members. Personal information in the protocol may be blacked out, with written motivations as footnotes.
  \item The original attested protocols shall be stored at a safe location by the board.
  \item The board can meet with shorter notice (less than four (4) days), or with less than half of its members, if all board members agree (including those that will not be present at the meeting). 
\end{enumerate}

\subsection{Board meeting rights}
\begin{enumerate}
  \item Each attending board member has one vote.
  \item Members of the board can attend, speak, vote and make proposals during the board meeting.
  \item Members of the association can attend the board meeting.
  \item Members of the association can attend, speak and make proposals during the board meeting if they are co-opted by the board.
  \item Other persons can attend the board meeting, if co-opted by the board.
  \item Other persons can attend, speak and make proposals during the board meeting if they are co-opted by the board.
  \item The board can request the meeting to be held behind closed doors until further notice, if all the attending members with voting rights agree by simple majority.
  \item A board meeting behind closed doors can only be attended by: 
  \begin{enumerate}
    \item The auditors 
    \item Board members 
    \item Persons who are allowed by the board to attend the meeting
  \end{enumerate}
\end{enumerate}

\section*{Representation}
\subsection{}
\begin{enumerate}
  \item The board or two members of the board in conjunction represent the association in legal matters.
  \item Members of the board represent the association on other occasions.
\end{enumerate}

\section*{Auditors }
\subsection{}
\begin{enumerate}
  \item The auditors are responsible for checking: 
  \begin{enumerate}
    \item That the board operates according to the operational plan, the statutes and the purpose of the association.
    \item That the bookkeeping is in order.
  \end{enumerate}
  \item The auditors are chosen at the fall meeting. There should be at least two auditors. 
  \item The auditors should have access to all documents of the board at their will.
  \item Investigations should be ongoing throughout the whole year.
  \item The auditors have the power to call for an extraordinary general meeting.
  \item The auditors shall prepare an audit report for the annual meeting. The audit report must contain a recommendation for discharge of liability of the board or not.
  \item Auditors cannot be members of the board. 
\end{enumerate}

\section*{Election Committee}
\subsection{}
\begin{enumerate}
  \item The election committee is responsible for making a proposal for a new board and auditors to be elected at the fall meeting.
  \item The election committee for the year is elected at the spring meeting.
\end{enumerate}

\section*{General Meeting}

\subsection{Right to make decisions}
The general meeting has the right to make decisions if and only if:
\begin{enumerate}
  \item An invitation to the meeting has been sent out (by email or postal mail) to all association members three weeks before the meeting. The invitation should contain a preliminary agenda.
  \item An extraordinary meeting can be called with one weeks’ notice. 
\end{enumerate}
\subsection{Meeting documents} \label{sec:regMeetingDocs}
\begin{enumerate}
  \item All members should have access to the meeting documents two days before the meeting. 
  \item The meeting documents should include the agenda of the meeting, motions and bills.
\end{enumerate}
\subsection{Decision making}
\begin{enumerate}
  \item Each present member has one vote during the meeting. Distance voting is not allowed.
  \item Suggestion to the general meeting are considered motions if they are sent to the board at least one week before the meeting. Motions from the board are considered bills.
  \item All other suggestions are to be considered any other business and can only be raised by a unanimous vote\footnote{If no-one objects the suggestion to be brought up.}.
  \item Decisions are made by simple majority vote if nothing else is written in the statutes. 
\end{enumerate}
\subsection{Protocol}
\begin{enumerate}
  \item The board has to make sure that a decision protocol is written during the meeting.
  \item A copy of the attested decision protocol should be accessible to all members.
  \item The originals (printed and signed versions) of the meeting documents shall be stored at a safe location by the board. 
\end{enumerate}

\section*{Fall Meeting}
\subsection{General}
\begin{enumerate}
  \item The fall meeting is a general meeting and thus follows the rules for general meetings.
  \item There should be a fall meeting each year.
  \item The fall meeting should be during November.
  \item The time and location is decided by the board.
  \item If a fall meeting has not been held by the end of November, an individual member can summon the meeting and demand meeting documents from the board.
  \item The fall meeting should address the topics of \ref{sec:fallAgenda}.
\end{enumerate}

\subsection{Meeting documents}
Additional things that should be included in meeting documents are:
\begin{enumerate}
  \item Election committee’s suggestion for new board.
  \item Suggestion for operational plan.
  \item Suggestion for membership fee.
  \item Suggestion for budget. 
\end{enumerate}

\section*{Spring Meeting}
\subsection{General}
\begin{enumerate}
  \item The spring meeting is a general meeting and thus follows the rules for general meetings.
  \item There should be a spring meeting each year.
  \item The spring meeting should be held in February.
  \item The time and location is decided by the board.
  \item If a spring meeting has not been held by the end of February, an individual member can summon the meeting and demand meeting documents from the board.
  \item The spring meeting should address the topics of \ref{sec:springAgenda}.
\end{enumerate}

\subsection{Meeting documents}
Additional things that should be included in meeting documents are:
\begin{enumerate}
  \item Activity report
  \item Bookkeeping
  \item Audit report
\end{enumerate}

\cleardoublepage
\section*{General meeting agendas}

\subsection{Agenda} \label{sec:fallAgenda}
The following matters should be addressed by the fall meeting:
\begin{enumerate}
  \item Formalities:
  \begin{enumerate}
    \item Opening of the meeting.
    \item Election of chairperson for the meeting.
    \item Election of secretary for the meeting.
    \item Election of two attestors to attest the protocol who are also tellers\footnote{Tellers have the role of counting votes during the meeting.} during the meeting.
    \item Creation of voting list.
    \item Verification of meetings right to make decisions.
    \item Establishing of agenda.
  \end{enumerate}
  \item Elections:
  \begin{enumerate}
    \item Election of new board and substitutes.
    \item Election of auditors and substitutes.
    \item Election of election committee and substitutes.
  \end{enumerate}
  \item The following year:
  \begin{enumerate}
    \item Determination of operational plan.
    \item Determination of membership fee.
    \item Determination of budget.
    \item Bills.
    \item Motions.
  \end{enumerate}
  \item Any other business
  \item Formalities:
  \begin{enumerate}
    \item Closure of meeting
  \end{enumerate}
\end{enumerate}
\clearpage

\subsection{Spring meeting agenda} \label{sec:springAgenda}
The following matters should be addressed by the spring meeting:
\begin{enumerate}
  \item Formalities:
  \begin{enumerate}
    \item Opening of the meeting.
    \item Election of chairperson for the meeting.
    \item Election of secretary for the meeting.
    \item Election of two attestors to attest the protocol who are also tellers during the meeting.
    \item Creation of voting list.
    \item Verification of meetings right to make decisions.
    \item Establishing of agenda.
  \end{enumerate}
  \item The previous year:
  \begin{enumerate}
    \item Presentation of activity report (by the board).
    \item Presentation of bookkeeping (by the board).
    \item Presentation of audit report (by the auditors). 
    \item Decision of discharge of liability of the board.
    \item Decision of discharge of liability of the auditors. 
  \end{enumerate}
  \item Suggestions:
  \begin{enumerate}
    \item Bills.
    \item Motions.
  \end{enumerate}
  \item Any other business
  \item Formalities:
  \begin{enumerate}
    \item Closure of meeting
  \end{enumerate}
\end{enumerate}
\cleardoublepage

\section*{Dissolving the Association}
\subsection{Procedure}
\begin{enumerate}
  \item The association can be dissolved by a two-thirds majority vote during two consecutive general meetings, of which one is the spring or fall meeting. A minimum of four (4) weeks must elapse between the two meetings.
  \item In order to dissolve the association, the request to do so must be included in the invitation for both of the meetings required.
\end{enumerate}

\subsection{Financial assets}
\begin{enumerate}
  \item Any left-over funding from outside sponsors will be returned.
  \item What happens to financial assets except left-over funding may be regulated in the house rules. Otherwise a plan for this must be included in the meeting documents for the two meetings required to dissolve the association.
\end{enumerate}