
\section*{About the Association}

\subsection{Name and location}
\begin{enumerate}
  \item The name of the association is “Association of Engineering Students in Rocketry”, abbreviated ÆSIR or AESIR.
  \item The Æ letter is written, if possible, using the unicode character \textbf{U+00C6} or the HTML entity \textbf{\&\#198;}.
  \item In legal matters, the non-abbreviated full name is used.
  \item The association is located in Stockholms kommun.
\end{enumerate}

\subsection{Principles}
\begin{enumerate}
  \item The association is a non-profit association.
  \item The association is politically and religiously impartial.
  \item The association works according to democratic principles.
  \item The core values of the association are \emph{collaboration}, \emph{education} and \emph{transparency}.
\end{enumerate}


\subsection{Year}
\begin{enumerate}
  \item The operational year runs from January 1st to December 31st.
  \item The fiscal year runs from January 1st to December 31st.
\end{enumerate}


\subsection{Authority}
\begin{enumerate}
  \item The general meeting is the highest governing body of the association.
  \item The board is the highest executive body of the association.
  \item Between general meetings, the board acts as the highest governing body of the association. 
\end{enumerate}

\section*{Purpose}

\subsection{Goals}
The association aims to:
\begin{enumerate}
  \item Design, build and launch rockets.
  \item Give members real world engineering experience in complex projects.
  \item Educate students in rocketry and engineering practices.
  \item Accumulate and maintain knowledge on rocketry and engineering practices.
\end{enumerate}

\subsection{Process}
The association seeks to achieve its objectives by: 
\begin{enumerate}
  \item Setting up projects to design, construct and launch rockets.
  \item Facilitating the construction and launching of rockets in legal and logistical matters.
  \item Providing a framework where members can educate themselves in engineering.
  \item Facilitating the arrangement of lectures, activities and company visits.
\end{enumerate}

\section*{Regulatory Documents}
\subsection{Availability}
\begin{enumerate}
  \item Copies of the current version of all regulatory documents should be stored at a location – physical or digital – such that all members of the association can access them.
  \item Signed originals of regulatory documents should be stored at a safe location by the board.
\end{enumerate}

\subsection{Interpretation}
\begin{enumerate}
  \item If there are any uncertainties with the interpretation of the statutes, they are interpreted by the general meeting.
  \item Outside of the general meeting, the statutes are interpreted by the board.
  \item An interpretation of the statutes outside a general meeting must be tested at the next general meeting.
\end{enumerate}

\subsection{Statutes}
\begin{enumerate}
  \item The statutes can be changed by decision of two consecutive general meetings, of which one is the spring or fall meeting. A minimum of four (4) weeks must elapse between the two meetings.
  \item In order to change the statutes, the request to do so and the proposed changes must be included in the invitation and the meeting documents for both of the meetings required to complete the process.   
\end{enumerate}

\subsection{House rules}
\begin{enumerate}
  \item The house rules regulate the operation of the association in more detail than the statutes. 
  \item The house rules are always subordinated to the statutes.
  \item The house rules can be changed by decision of the general meeting, as long as the request to do so and a document of the proposed changes is properly sent as specified in \ref{sec:regMeetingDocs}.
\end{enumerate}

\section*{Membership}
\subsection{Definition}
A member of the association is someone who has initiated membership according to~\ref{sec:membershipInitiation} 
and has not terminated their membership according to~\ref{sec:membershipTermination}.

\subsection{Initiation of membership}\label{sec:membershipInitiation}
A person can become a member of the association once the following criteria are met:
\begin{enumerate}
  \item They have signed the membership contract. % Does this reference something useful?
  \item They have paid the membership fee for the current membership period (as defined in~\ref{sec:membershipPeriod}).
\end{enumerate}

\subsection{Period} \label{sec:membershipPeriod}
The membership period is either the spring period or the fall period, each spanning half of one calendar year:

\begin{enumerate}
  \item The spring period runs from the 1st of January until the 30th of June.
  \item The fall period runs from the 1st of July until the 31st of December.
\end{enumerate}

\subsection{Rights}
Members have the right to: 
\begin{enumerate}
  \item Attend, speak and vote during general meetings.
  \item Attend board meetings.
  \item Have access to current attested versions of the regulatory documents of the association.
  \item Have access to current source documents of the regulatory documents of the association.
  \item Have access to attested protocols from board meetings and general meetings.
\end{enumerate}

\subsection{Responsibilities}
\begin{enumerate}
  \item Members have to pay the membership fee in accordance to~\ref{sec:membershipFee}.
  \item Members need to follow the house rules, if any are currently in effect.
\end{enumerate}

\subsection{Membership fee} \label{sec:membershipFee}
\begin{enumerate}
  \item Members have to pay the membership fee for each membership period.
  \item Members can pay for both the current spring period and the upcoming fall period if currently in the spring period.
  \item Members cannot pay for the upcoming spring and fall periods before the membership fee for the coming year has been determined by the fall meeting. \label{sec:feeYear}
  \item However if the fee for the coming year has not yet been determined by the new year, it remains as it was when it was last determined, and item~\ref{sec:feeYear} does not then apply.
  \item The membership fee should be divisible with 2 and is split evenly among the spring and fall periods.
\end{enumerate}

\subsection{Termination of membership} \label{sec:membershipTermination}
Membership is terminated:
\begin{enumerate}
  \item Automatically, if the member has not paid the membership fee for the current period within a calendar month of the start of the period. Specifically, membership is terminated if:
  \begin{enumerate}
    \item The spring period fee is not paid before or on the 31st of January.
    \item The fall period fee is not paid before or on the 31st of July.
  \end{enumerate}
  \item If the member requests it to the board.
  \item Automatically, if the general meeting expels a member with a two-thirds majority vote. Repeals can only be made through decision by the general meeting.
\end{enumerate}

\section*{Board}
\subsection{Responsibilities}
The board is responsible for:
\begin{enumerate}
  \item The operation of the association.
  \item Making sure that decisions made during general meetings are executed.
  \item Summoning the fall and spring meetings. 
  \item Maintaining the finances of the association. 
\end{enumerate}

\subsection{Board miscellaneous}
\begin{enumerate}
  \item The board consists of at least three (3) and at most eight (8) members.
  \item There can be zero or more substitutes to the board.
  \item The board should at least have a president, a treasurer and a secretary.
  \item The members of the board are individually chosen in function by the general meeting.
  \item Members of the board should be members of the association. If a member of the board is no longer a member of the association, their role will be expired.
  \item The board is elected on the fall meeting, but is changed on the 1st of January the following year.
  \item Vacant board positions can be filled at the spring meeting.
  \item If there are less than three members of the board, an extraordinary general meeting needs to be held to fill the vacant positions.
\end{enumerate}

\subsection{Board meetings}
\begin{enumerate}
  \item The board has to meet at least once every quarter of the year, or if at least a third of the board members have made a written request to the president of the association.
  \item The president of the association should notify the board members about the meeting at least four (4) days in advance.
  \item The board can only make decisions if at least half of the board members are present and under the condition that the meeting is summoned in the correct way.
  \item Decisions during the board's regular meetings are made by simple majority.
  \item The board has to write decision protocols at its meetings. Personal information in the protocol may be blacked out, with written motivations as footnotes.
  \item Board meeting protocols should be attested within fourteen (14) days of the meeting.
  \item Scanned attested board meeting protocols should be available to all members of the association online within twentyone (21) days of the meeting.
  \item The original attested protocols shall be stored at a safe location by the board.
  \item The board can meet with shorter notice (less than four (4) days), or with less than half of its members, if all board members agree (including those that will not be present at the meeting). 
\end{enumerate}

\subsection{Board meeting rights}
\begin{enumerate}
  \item Each attending board member has one vote.
  \item Members of the board can attend, speak, vote and make proposals during the board meeting.
  \item Members of the association can attend the board meeting.
  \item Members of the association can attend, speak and make proposals during the board meeting if they are co-opted by the board.
  \item Other persons can attend the board meeting, if co-opted by the board.
  \item Other persons can attend, speak and make proposals during the board meeting if they are co-opted by the board.
  \item The board can request the meeting to be held behind closed doors until further notice, if all the attending members with voting rights agree by simple majority.
  \item A board meeting behind closed doors can only be attended by: 
  \begin{enumerate}
    \item The auditors 
    \item Board members 
    \item Persons who are allowed by the board to attend the meeting
  \end{enumerate}
\end{enumerate}

\section*{Representation}
\subsection{}
\begin{enumerate}
  \item The board or two members of the board in conjunction represent the association in legal matters.
  \item Members of the board represent the association on other occasions.
\end{enumerate}

\section*{Auditors}
\subsection{}
\begin{enumerate}
  \item The auditors are responsible for checking: 
  \begin{enumerate}
    \item That the board operates according to the operational plan, the statutes and the purpose of the association.
    \item That the bookkeeping is in order.
  \end{enumerate}
  \item The auditors are elected at the fall meeting. 
  \item There should be at least two auditors. 
  \item There can be zero or more substitutes to the auditors.
  \item The auditors should have access to all documents of the board at their will.
  \item Auditors cannot be members of the board. 
  \item Auditors must be resident in Sweden. If an auditor is no longer resident in Sweden, their role will be expired.
  \item The auditors are elected in conjunction with a board.
  \item \textit{The audited board} refers to the board which the auditors are elected in conjunction with.
\end{enumerate}

\subsection{Tasks}
The work of the auditors is split up into two parts:
\begin{enumerate}
  \item They should do work throughout the year they are elected for as noted in~\ref{sec:auditorYear}.
  \item They should create a final report and recommendation to the spring meeting during the year \textit{after} their elected year has ended as noted in~\ref{sec:finalReport}.
\end{enumerate}

\subsection{Work throughout the year} \label{sec:auditorYear}
During the year which the auditors are elected for, their work should involve:
\begin{enumerate}
  \item Investigations which should be ongoing throughout the whole year.
  \item Call for an extraordinary general meeting if necessary.
\end{enumerate}

\subsection{Final report and recommendation} \label{sec:finalReport}
After the year which the auditors are elected for, they should present an audit report on the next spring meeting:
\begin{enumerate}
  \item The auditors shall prepare an audit report, summarizing the year they were elected for, for the spring meeting. 
  \item The audit report must contain a recommendation or non-recommendation for discharge of liability of the audited board.
\end{enumerate}

\subsection{Audit report}
The audit report should contain:
\begin{enumerate}
  \item A presentation of whether the bookkeeping is in order.
  \item A presentation of anomalies (if applicable).
  \item A presentation of investigations that has been undertaken.
\end{enumerate}

\subsection{Recommendation or non-recommendation for discharge of liability}
The auditors present a recommendation or non-recommendation for discharge of liability of the audited board they were elected with.
This recommendation can be brief, but should be based on at least:
\begin{enumerate}
  \item If the bookkeeping is not in order: do not recommend discharge of liability.
  \item If anomalies are not resolved: do not recommend discharge of liability.
  \item If investigations have uncovered unresolved issues: do not recommend discharge of liability.
\end{enumerate}

\section*{Election Committee}
\subsection{}
\begin{enumerate}
  \item The election committee is responsible for making a proposal for roles to be elected at the fall meeting. These roles are:
  \begin{enumerate}
    \item Board and substitutes
    \item Auditors and substitutes
    \item Election committee and substitutes
  \end{enumerate}
  \item Members of the election committee must be members of the association. If a member of the election committee is no longer a member of the association, their role will be expired.
  \item The election committee for the year is elected at the fall meeting.
  \item There should be at least two members of the election committee. 
  \item There can be zero or more substitutes to the election committee.
  \item If the association does not have an election committee, the auditors act as substitutes.
  \item If the auditors cannot act as substitutes, the board will act as substitutes.
\end{enumerate}

\section*{General Meeting}

\subsection{Right to make decisions}
The general meeting has the right to make decisions if and only if:
\begin{enumerate}
  \item The meeting invitation is in accordance with~\ref{sec:meetingInvitation}.
  \item The meeting documents are in accordance with~\ref{sec:regMeetingDocs}.
\end{enumerate}

\subsection{Invitation} \label{sec:meetingInvitation}
\begin{enumerate}
  \item An invitation to the fall or spring meeting must be sent out (by email or postal mail) to all members at least (21) days before the meeting.
  \item An extraordinary meeting can be called seven (7) days in advance. 
  \item The invitation must include:
  \begin{enumerate}
    \item A preliminary meeting agenda.
    \item All proposed changes to the statutes.
  \end{enumerate}
\end{enumerate}

\subsection{Meeting documents} \label{sec:regMeetingDocs}
\begin{enumerate}
  \item All members should have access to the meeting documents at least fourty-eight (48) hours before the meeting. 
  \item The meeting documents must include:
  \begin{enumerate}
    \item The proposed agenda of the meeting.
    \item The motions and bills.
    \item All proposed changes to the statutes, as included with the invitation.
  \end{enumerate}
\end{enumerate}

\subsection{Decision making}
\begin{enumerate}
  \item Each present member has one vote during the meeting. Distance voting is not allowed.
  \item Suggestion to the general meeting are considered motions if they are sent to the board at least one week before the meeting. Motions from the board are considered bills.
  \item All other suggestions are to be considered any other business and can only be raised by a unanimous vote.
  \item Decisions are made by simple majority vote if nothing else is written in the statutes. 
\end{enumerate}
\subsection{Protocol}
\begin{enumerate}
  \item The board has to make sure that a decision protocol is written during the meeting.
  \item The general meeting protocol should include all meeting documents.
  \item Scanned and attested general meeting protocols should be available to all members of the association online within fourteen (14) days of the meeting.
  \item General meeting protocols should be attested within seven (7) days of the meeting.
  \item The originals (attested versions) of the meeting documents shall be stored at a safe location by the board. 
\end{enumerate}

\section*{Extraordinary Meeting}
\subsection{General}
\begin{enumerate}
  \item An extraordinary meeting is a general meeting and thus follows the rules for general meetings.
  \item An extraordinary meeting cannot be either the fall or spring meeting.
  \item The time and location is decided by the one who calls for the meeting.
\end{enumerate}

\section*{Fall Meeting}
\subsection{General}
\begin{enumerate}
  \item The fall meeting is a general meeting and thus follows the rules for general meetings.
  \item There should be a fall meeting each year.
  \item The time and location is decided by the board.
  \item The fall meeting should be during November.
  \item If a fall meeting has not been held by the end of November, an individual member can summon the meeting and demand meeting documents from the board.
  \item The fall meeting should address the topics of \ref{sec:fallAgenda}.
\end{enumerate}

\subsection{Meeting documents}
Additional things that should be included in meeting documents are:
\begin{enumerate}
  \item Election committee’s suggestion for new board.
  \item Suggestion for operational plan.
  \item Suggestion for membership fee.
  \item Suggestion for budget. 
\end{enumerate}

\section*{Spring Meeting}
\subsection{General}
\begin{enumerate}
  \item The spring meeting is a general meeting and thus follows the rules for general meetings.
  \item There should be a spring meeting each year.
  \item The time and location is decided by the board.
  \item The spring meeting should be held in April.
  \item If a spring meeting has not been held by the end of April, an individual member can summon the meeting and demand meeting documents from the board.
  \item The spring meeting should address the topics of \ref{sec:springAgenda}.
\end{enumerate}

\subsection{Meeting documents}
Additional things that should be included in meeting documents are:
\begin{enumerate}
  \item Activity report
  \item Bookkeeping
  \item Audit report
\end{enumerate}

\section*{Dissolving the Association}
\subsection{Procedure}
\begin{enumerate}
  \item The association can be dissolved by a two-thirds majority vote during two consecutive general meetings, of which one is the spring or fall meeting. A minimum of four (4) weeks must elapse between the two meetings.
  \item In order to dissolve the association, the request to do so must be included in the invitation for both of the meetings required.
\end{enumerate}

\subsection{Financial assets}
\begin{enumerate}
  \item Any left-over funding from outside sponsors will be returned.
  \item What happens to financial assets except left-over funding may be regulated in the house rules. Otherwise a plan for this must be included in the meeting documents for the two meetings required to dissolve the association.
\end{enumerate}

\cleardoublepage
\section*{General meeting agendas}

\subsection{Agenda} \label{sec:fallAgenda}
The following matters should be addressed by the fall meeting:
\begin{enumerate}
  \item Formalities:
  \begin{enumerate}
    \item Opening of the meeting.
    \item Election of chairperson for the meeting.
    \item Election of secretary for the meeting.
    \item Election of two attestors to attest the protocol who are also tellers\footnote{Tellers have the role of counting votes during the meeting.} during the meeting.
    \item Creation of voting list.
    \item Verification of meetings right to make decisions.
    \item Establishing of agenda.
  \end{enumerate}
  \item Elections:
  \begin{enumerate}
    \item Election of new board and substitutes.
    \item Election of auditors and substitutes.
    \item Election of election committee and substitutes.
  \end{enumerate}
  \item The following year:
  \begin{enumerate}
    \item Determination of operational plan.
    \item Determination of membership fee.
    \item Determination of budget.
    \item Bills.
    \item Motions.
  \end{enumerate}
  \item Any other business
  \item Formalities:
  \begin{enumerate}
    \item Closure of meeting
  \end{enumerate}
\end{enumerate}
\clearpage

\subsection{Spring meeting agenda} \label{sec:springAgenda}
The following matters should be addressed by the spring meeting:
\begin{enumerate}
  \item Formalities:
  \begin{enumerate}
    \item Opening of the meeting.
    \item Election of chairperson for the meeting.
    \item Election of secretary for the meeting.
    \item Election of two attestors to attest the protocol who are also tellers during the meeting.
    \item Creation of voting list.
    \item Verification of meetings right to make decisions.
    \item Establishing of agenda.
  \end{enumerate}
  \item The previous year:
  \begin{enumerate}
    \item Presentation of activity report (by the board).
    \item Presentation of bookkeeping (by the board).
    \item Presentation of audit report (by the auditors). 
    \item Decision of discharge of liability of the board.
    \item Decision of discharge of liability of the auditors. 
  \end{enumerate}
  \item Suggestions:
  \begin{enumerate}
    \item Bills.
    \item Motions.
  \end{enumerate}
  \item Any other business
  \item Formalities:
  \begin{enumerate}
    \item Closure of meeting
  \end{enumerate}
\end{enumerate}
\cleardoublepage
